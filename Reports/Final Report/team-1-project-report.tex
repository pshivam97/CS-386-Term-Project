% Acknowledgements 
% based on https://www.overleaf.com/latex/templates/ece-100-template

% DO NOT CHANGE THIS PART !!!!!!!!!!!!!
\documentclass[a4paper, 10pt,twocolumn]{article}
\usepackage{geometry}
\usepackage{comment} % for multi-line comments (\ifx \fi) 
\usepackage{lipsum} % generates Lorem Ipsum filler text. 
\usepackage{newtxtext}
\usepackage{amsmath}
\usepackage{amsfonts}
\usepackage{amsthm}
\usepackage{mathtools}
\usepackage{bm}
\usepackage{graphicx,psfrag,epsf}
\usepackage{subfigure}
\geometry{margin=0.6in}
% DO NOT CHANGE THIS PART !!!!!!!!!!!!!

\begin{document}
	
\title{Single-Line Handwritten Sentence Recognition}
\author{Shivam Pandey | Shivam Kumar | Abhishek Varghese\\Team 1} 
\date{\today}

\maketitle

% NOTE: NOT MORE THAN 2 PAGES  

 
\section{Introduction and Overview}
Over the last few decades, research on handwriting recognition has made impressive progress. The research and development on handwritten word recognition are to a large degree motivated by many application areas, such as automated postal address, data acquisition in banks, text-voice conversion, security, etc. As the prices of scanners, computers and handwriting-input devices are falling steadily, we have seen an increased demand for handwriting recognition systems and software packages. Few commercial handwriting recognition systems are now available in the market. Current commercial systems have an impressive performance in recognizing machine-printed characters and neatly written texts. For instance, Xerox in the U.S. has developed TextBridge for converting hardcopy documents into electronic document files. Notwithstanding the impressive progress, there is still a significant performance gap between the human and the machine in recognizing off-line unconstrained handwritten words. The difficulties encountered in recognizing unconstrained handwritings are mainly caused by huge variations in writing styles and the overlapping and the interconnection of neighboring characters. Furthermore, many applications demand very high recognition accuracy and reliability. Sadly, in the banking sector, although ATMs and networked banking systems being widely available, many transactions are still carried out in the form of cheques.\emph{Google's Tesseract}, despite being developed by the greatest IT beast, its output will have very poor quality if the input images are not preprocessed to suit it. This motivates us to research and develop a handwritten sentence recognizer with much improved accuracy.
\subsection*{Related work}

Discuss about any prior/similar work. 

Example of a Citation\cite[p.219]{Robotics}. Here's 
Another Citation\cite{Flueck} % Make sure to remove this 

\section{Methods}
We have exploited \emph{deep learning} methods such as \emph{Convolutional Neural Network (CNN), Recurrent Neural Network (RNN), Long Short Term Memory (LSTM) units} and \emph{Connectionist Temporal Classification (CTC)}. The input given to the recognition system is an image of a single-line handwritten English sentence, which is processed through a series of aforementioned deep learning techniques to finally output the sentence in machine-printed format.

\section{Experimental Analyses} 

\subsection*{Datasets} 
\subsection*{Results} 

\section{Discussion and Future Directions} 

\begin{thebibliography}{9}
	
 % Make sure to remove this and add relevant references 

\bibitem{Robotics} Fred G. Martin \emph{Robotics Explorations: A 
Hands-On Introduction to Engineering}. New Jersey: Prentice Hall.

\bibitem{Flueck}  Flueck, Alexander J. 2005. \emph{ECE 
100}[online]. Chicago: Illinois Institute of Technology, Electrical 
and Computer Engineering Department, 2005 [cited 30
August 2005]. Available from World Wide Web: (http://www.ece.iit.edu/~flueck/ece100).

\end{thebibliography}

\end{document}