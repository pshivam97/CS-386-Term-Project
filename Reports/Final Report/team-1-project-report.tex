% Acknowledgements 
% based on https://www.overleaf.com/latex/templates/ece-100-template

% DO NOT CHANGE THIS PART !!!!!!!!!!!!!
\documentclass[a4paper, 10pt,twocolumn]{article}
\usepackage{geometry}
\usepackage{comment} % for multi-line comments (\ifx \fi) 
\usepackage{lipsum} % generates Lorem Ipsum filler text. 
\usepackage{newtxtext}
\usepackage{amsmath}
\usepackage{amsfonts}
\usepackage{amsthm}
\usepackage{mathtools}
\usepackage{bm}
\usepackage{graphicx,psfrag,epsf}
\usepackage{subfigure}
\geometry{margin=0.6in}
% DO NOT CHANGE THIS PART !!!!!!!!!!!!!

\begin{document}
	
\title{Project Title}
\author{Team members\\Team XX} 
\date{\today}

\maketitle

% NOTE: NOT MORE THAN 2 PAGES  

 
\section{Introduction and Overview}

Give an introduction to your project. 


\subsection*{Related work}

Discuss about any prior/similar work. 

Example of a Citation\cite[p.219]{Robotics}. Here's 
Another Citation\cite{Flueck} % Make sure to remove this 



\section{Methods}

Discuss about any AI methods used by team 


\section{Experimental Analyses} 

\subsection*{Datasets} 
\subsection*{Results} 

\section{Discussion and Future Directions} 

\begin{thebibliography}{9}
	
 % Make sure to remove this and add relevant references 

\bibitem{Robotics} Fred G. Martin \emph{Robotics Explorations: A 
Hands-On Introduction to Engineering}. New Jersey: Prentice Hall.

\bibitem{Flueck}  Flueck, Alexander J. 2005. \emph{ECE 
100}[online]. Chicago: Illinois Institute of Technology, Electrical 
and Computer Engineering Department, 2005 [cited 30
August 2005]. Available from World Wide Web: (http://www.ece.iit.edu/~flueck/ece100).

\end{thebibliography}

\end{document}